\section{Cải tiến Mô hình và Tối ưu hóa Kỹ thuật}\label{sec:model-improvements}

\noindent
Phần này tập trung vào các \textbf{cải tiến kỹ thuật} và \textbf{tối ưu hóa} được triển khai trong dự án AIO Classifier. Bao gồm việc cải tiến thuật toán, tối ưu hóa hiệu suất, tích hợp hệ thống và các chiến lược implementation hiện đại.

\subsection{Tối ưu hóa Thuật toán Cơ bản}\label{subsec:basic-algorithm-optimization}

\subsubsection{Cải tiến Tree-Based Models}

\textbf{Random Forest Optimizations}:
\begin{itemize}
    \item \textbf{Tinh chỉnh Tham số}: Đã tối ưu hóa các tham số cốt lõi như số lượng cây decision tree (n\_estimators), độ sâu tối đa của cây (max\_depth), và số lượng mẫu tối thiểu để phân chia (min\_samples\_split) cho dữ liệu tim mạch
    \item \textbf{Tầm quan trọng Đặc trưng}: Xây dựng hệ thống tính toán tầm quan trọng của từng đặc trưng y khoa (tuổi, giới tính, đau ngực...) được xác thực dựa trên phương pháp permutation để đảm bảo độ tin cậy về mặt lâm sàng
    \item \textbf{Lấy mẫu Bootstrap}: Cải thiện phương pháp bootstrap sampling với phân phối cân bằng giữa nhóm có bệnh tim và nhóm khỏe mạnh để tránh thiên lệch về phía nhóm đa số
    \item \textbf{Tích hợp Cache}: Tối ưu hóa hệ thống cache cho việc lưu trữ các cây decision tree đã được huấn luyện để tăng tốc quá trình dự đoán trong thời gian thực
\end{itemize}

\textbf{Cải tiến Gradient Boosting}:
\begin{itemize}
    \item \textbf{Tối ưu hóa Tốc độ Học}: Triển khai lập trình thích ứng tốc độ học (adaptive learning rate scheduling) để ngăn ngừa hiện tượng overfitting - khi mô hình học quá kỹ trên dữ liệu huấn luyện nhưng kém khả năng tổng quát hóa
    \item \textbf{Dừng Sớm Thông Minh}: Triển khai thuật toán early stopping thông minh với việc theo dõi tập validation để tự động dừng huấn luyện khi hiệu suất không còn cải thiện, tiết kiệm thời gian và tài nguyên tính toán
    \item \textbf{Lựa chọn Đặc trững}: Tích hợp thuật toán lựa chọn đặc trưng tự động trong quá trình boosting để chỉ giữ lại những đặc trưng quan trọng nhất cho dự đoán tim mạch, giảm noise và nâng cao hiệu suất
    \item \textbf{Hiệu quả Bộ nhớ}: Tối ưu hóa việc sử dụng bộ nhớ cho các tác vụ boosting quy mô lớn thông qua techniques như histogram binning và sparse matrix operations để có thể xử lý datasets tim mạch lớn hơn
\end{itemize}

\subsubsection{Linear Model Enhancements}

\textbf{Tối ưu hóa Logistic Regression}:
\begin{itemize}
    \item \textbf{Lựa chọn Thuật toán Giải}: Tự động lựa chọn thuật toán giải tối ưu (L-BFGS) cho bài toán phân loại đa lớp trong dữ liệu y khoa tim mạch, cân bằng giữa tốc độ tính toán và độ chính xác
    \item \textbf{Điều chính Regularization}: Tối ưu hóa tham số C để cân bằng bias-variance tradeoff - đảm bảo mô hình không quá đơn giản (underfitting) hay quá phức tạp (overfitting) khi dự đoán bệnh tim mạch
    \item \textbf{Hỗ trợ Đa luồng}: Nâng cấp hỗ trợ tính toán đa luồng để tăng tốc quá trình hội tụ của thuật toán, đặc biệt quan trọng khi xử lý datasets lớn và cần dự đoán nhanh trong môi trường lâm sàng
    \item \textbf{Hỗ trợ Ma trận Sparse}: Tối ưu hóa các thao tác ma trận sparse để tăng hiệu quả sử dụng bộ nhớ khi xử lý dữ liệu y khoa có nhiều thuộc tính bằng 0 hoặc thiếu dữ liệu (missing values)
\end{itemize}

\textbf{Cải tiến SVM}:
\begin{itemize}
    \item \textbf{Lựa chọn Kernel Tự động}: Hệ thống tự động chọn loại kernel phù hợp (RBF, Polynomial, Linear) dựa trên đặc điểm của dữ liệu tim mạch - để phân loại patterns phi tuyến trong triệu chứng bệnh tim
    \item \textbf{Tối ưu hóa Scaler Sensitivity}: Phát triển chiến lược tối ưu hóa đặc biệt cho các phương pháp chuẩn hóa dữ liệu khác nhau (StandardScaler, MinMaxScaler, RobustScaler) vì SVM rất nhạy cảm với preprocessing
    \item \textbf{Quản lý Bộ nhớ Kernel}: Tối ưu hóa quản lý bộ nhớ cho các ma trận kernel để xử lý hiệu quả datasets lớn mà không gây tràn bộ nhớ khi tính toán khoảng cách giữa các điểm dữ liệu
    \item \textbf{Tối ưu hóa Pipeline Dự đoán}: Xây dựng pipeline dự đoán được tối ưu hóa để tăng tốc inference trong môi trường production - quan trọng cho ứng dụng lâm sàng thời gian thực
\end{itemize}

\subsection{Cải tiến Cụ thể Thuật toán}\label{subsec:algorithm-specific}

\subsubsection{Tăng cường Thuật toán Cây quyết định}

\textbf{Random Forest Optimizations}:
\begin{itemize}
    \item \textbf{Tinh chỉnh Tham số}: Đã tối ưu hóa các tham số cốt lõi như số lượng cây decision tree (n\_estimators), độ sâu tối đa của cây (max\_depth), và số lượng mẫu tối thiểu để phân chia (min\_samples\_split) cho dữ liệu tim mạch
    \item \textbf{Tầm quan trọng Đặc trưng}: Xây dựng hệ thống tính toán tầm quan trọng của từng đặc trưng y khoa (tuổi, giới tính, đau ngực...) được xác thực dựa trên phương pháp permutation để đảm bảo độ tin cậy về mặt lâm sàng
    \item \textbf{Lấy mẫu Bootstrap}: Cải thiện phương pháp bootstrap sampling với phân phối cân bằng giữa nhóm có bệnh tim và nhóm khỏe mạnh để tránh thiên lệch về phía nhóm đa số
    \item \textbf{Tích hợp Cache}: Tối ưu hóa hệ thống cache cho việc lưu trữ các cây decision tree đã được huấn luyện để tăng tốc quá trình dự đoán trong thời gian thực
\end{itemize}

\textbf{Cải tiến Gradient Boosting}:
\begin{itemize}
    \item \textbf{Tối ưu hóa Tốc độ Học}: Triển khai lập trình thích ứng tốc độ học (adaptive learning rate scheduling) để ngăn ngừa hiện tượng overfitting - khi mô hình học quá kỹ trên dữ liệu huấn luyện nhưng kém khả năng tổng quát hóa
    \item \textbf{Dừng Sớm Thông Minh}: Triển khai thuật toán early stopping thông minh với việc theo dõi tập validation để tự động dừng huấn luyện khi hiệu suất không còn cải thiện, tiết kiệm thời gian và tài nguyên tính toán
    \item \textbf{Lựa chọn Đặc trững}: Tích hợp thuật toán lựa chọn đặc trưng tự động trong quá trình boosting để chỉ giữ lại những đặc trưng quan trọng nhất cho dự đoán tim mạch, giảm noise và nâng cao hiệu suất
    \item \textbf{Hiệu quả Bộ nhớ}: Tối ưu hóa việc sử dụng bộ nhớ cho các tác vụ boosting quy mô lớn thông qua techniques như histogram binning và sparse matrix operations để có thể xử lý datasets tim mạch lớn hơn
\end{itemize}

\subsection{Phân tích Phương pháp Ensemble}\label{subsec:ensemble-methods}  

\subsubsection{Hiệu suất Voting Ensemble}

\textbf{Cấu hình Hard Voting}:
\begin{itemize}
    \item \textbf{Lựa chọn Mô hình Cơ sở}: Đã tối ưu hóa việc lựa chọn các mô hình cơ sở (như Random Forest, SVM, Logistic Regression) với các điểm mạnh bổ sung nhau để tổng hợp dự đoán thông qua majority voting
    \item \textbf{Biến động Hiệu suất}: Hiệu suất thay đổi theo các phương pháp chuẩn hóa khác nhau (96.1-98.1\% độ chính xác), cho thấy robustless tốt với preprocessing variations
    \item \textbf{Hiệu quả Huấn luyện}: Thời gian huấn luyện cạnh tranh 5.92-6.34 giây, phù hợp cho triển khai production và cần thay đổi mô hình nhanh chóng
    \item \textbf{Tính Robust}: Khả năng tổng quát hóa tốt trên các phương pháp preprocessing khác nhau, quan trọng khi triển khai trong môi trường y tế đa dạng
\end{itemize}

\subsubsection{Tuyệt vời của Stacking Ensemble}

\textbf{Logistic Regression Meta-learner}:
\begin{itemize}
    \item \textbf{Hiệu suất Hoàn hảo}: Độ chính xác 100\% trên tất cả các phương pháp chuẩn hóa dữ liệu - đạt được performance cao nhất có thể trong bài toán phân loại tim mạch
    \item \textbf{Khả năng Meta-học}: Logistic regression làm meta-learner thể hiện khả năng học tập đặc biệt từ kết quả của các base models (Random Forest, SVM, Gradient Boosting...) để tổng hợp thành dự đoán cuối cùng  
    \item \textbf{Độ phức tạp Huấn luyện}: Thời gian huấn luyện lâu hơn 22.8-23.8 giây do cần trải qua 2 bước: huấn luyện base models trước, sau đó huấn luyện meta-learner trên predictions của base models
    \item \textbf{Chiến lược Tối ưu hóa Phân tầng}: Áp dụng optimization hierarchical - tối ưu từng base model một cách riêng lẻ, sau đó tối ưu meta-learner để đạt tổng thể performance cao nhất
\end{itemize}

\subsection{Tối ưu hóa Thời gian Training}\label{subsec:training-optimization}

\subsubsection{Phân tích Cấp độ Tốc độ}

\textbf{Mô hình Siêu nhanh (< 0.1 giây)}:
\begin{itemize}
    \item \textbf{Naive Bayes}: 0.015-0.019 giây với độ chính xác 82.5\% - Lý tưởng cho các ứng dụng cần dự đoán instant, tuy độ chính xác chỉ ở mức trung bình
    \item \textbf{Decision Tree}: 0.027-0.034 giây với độ chính xác 98.0-99.0\% - Đây là mô hình có balance tốt nhất giữa tốc độ và độ chính xác, lý tưởng cho triển khai production
    \item \textbf{SVM}: 0.029-0.035 giây với độ chính xác 76.4-80.6\% - Tốc độ tốt nhưng hiệu suất phụ thuộc nhiều vào phương pháp chuẩn hóa dữ liệu
\end{itemize}

\textbf{Mô hình Nhanh (0.1-1 giây)}:
\begin{itemize}
    \item \textbf{Logistic Regression}: 0.047-0.095 giây với độ chính xác 80.4-81.4\% - Mô hình tuyến tính đơn giản, phù hợp cho baseline và quick prototyping
    \item \textbf{KNN}: 0.055-0.061 giây với độ chính xác 82.5-84.5\% - Thuật toán dựa trên khoảng cách Euclidean, hiệu quả với datasets nhỏ-trung bình
    \item \textbf{AdaBoost}: 1.248-1.334 giây với độ chính xác 85.4\% - Ensemble method với sequential learning, moderate accuracy nhưng cần thời gian training dài hơn
\end{itemize}

\textbf{Mô hình Tốc độ Trung bình (1-10 giây)}:
\begin{itemize}
    \item \textbf{Random Forest}: 3.17-3.73 giây với độ chính xác 100\% - Balance tuyệt vời giữa speed và accuracy, leader trong category này
    \item \textbf{Gradient Boosting}: 3.92-3.96 giây với độ chính xác 100\% - Ensemble method mạnh, cần thời gian training lâu hơn Random Forest  
    \item \textbf{XGBoost}: 4.15-4.69 giây với độ chính xác 96.1\% - Gradient boosting optimization với regularization tốt nhưng accuracy thấp hơn slightly
    \item \textbf{LightGBM}: 6.19-6.34 giây với độ chính xác 100\% - Memory-efficient gradient boosting với leaf-wise tree building
\end{itemize}

\textbf{Mô hình Hiệu suất Cao - Tải nặng (> 10 giây)}:
\begin{itemize}
    \item \textbf{CatBoost}: 19.6-19.9 giây với độ chính xác 100\% - CatBoost tự động xử lý categorical features và đạt performance tối đa, nhưng chậm nhất
    \item \textbf{Stacking Ensemble}: 22.8-23.8 giây với độ chính xác 100\% - Meta-learning với 2-stage optimization: train base models → train meta-learner trên predictions
\end{itemize}

\subsection{Hiểu biết Hiệu suất So sánh}\label{subsec:comparative-insights}

\subsubsection{So sánh Gia đình Thuật toán}

\textbf{Thế mạnh của Gia đình Tree-Based}:
\begin{itemize}
    \item \textbf{Random Forest}: Hiệu suất tổng thể tốt nhất (100\% accuracy, training nhanh) - Ưu thế của bootstrap aggregation và random feature selection
    \item \textbf{Gradient Boosting}: Hiệu suất xuất sắc với balance tốt giữa tốc độ và độ chính xác - Sequential learning từ residuals để cải thiện dần dần
    \item \textbf{Decision Tree}: Training nhanh nhất với độ chính xác gần hoàn hảo - Đơn giản nhưng hiệu quả cao cho bài toán tim mạch binary classification
    \item \textbf{XGBoost/LightGBM}: Boosting hiện đại nhất với khả năng tổng quát hóa đặc biệt - Regularization và optimization techniques tiên tiến
\end{itemize}

\textbf{Hiệu suất Mô hình Tuyến tính}:
\begin{itemize}
    \item \textbf{Logistic Regression}: Hiệu suất trung bình với training rất nhanh - Đơn giản, robust, và lý tưởng như baseline model cho các comparison
    \item \textbf{SVM}: Hiệu suất thay đổi tùy theo phương pháp scaling được chọn - Nhạy cảm với preprocessing nhưng có khả năng handling non-linear relationships tốt với kernel tricks
    \item \textbf{Tối ưu hóa Chuyên biệt}: Các mô hình tuyến tính được lợi ích đáng kể từ việc preprocessing đúng cách (feature scaling, normalization)
\end{itemize}

\textbf{Sự ưu việt của Ensemble Methods}:
\begin{itemize}
    \item \textbf{Voting Ensembles}: Hiệu suất tốt với độ phức tạp vừa phải - Majority voting từ multiple base models để giảm variance và tăng robustness
    \item \textbf{Stacking Ensembles}: Hiệu suất hoàn hảo thông qua meta-learning với độ phức tạp cao hơn - Meta-learner học cách tổng hợp predictions từ base models
    \item \textbf{Hiệu quả Meta-learner}: Logistic regression tỏ ra cực kỳ hiệu quả làm meta-learner vì khả năng classification tốt và stability cao
\end{itemize}

\subsubsection{Mối tương quan Hiệu ứng Scaler}

\textbf{Khả năng Mở rộng của High Performers}:
\begin{itemize}
    \item \textbf{Tier 1 Models}: Hiệu suất xuất sắc nhất quán trên tất cả các scalers (Robust performance) - Tree-based models như Random Forest, CatBoost, LightGBM có ít sensitive với preprocessing
    \item \textbf{Tier 2 Models}: Hiệu suất tốt với độ nhạy cảm scaler nhẹ - Gradient Boosting và Decision Tree có slight variations nhưng overall stable  
    \item \textbf{Tier 3 Models}: Hiệu suất trung bình với độ nhạy cảm scaler rõ ràng - SVM và Logistic Regression performances vary significantly với different scalers
\end{itemize}

\textbf{Tối ưu hóa Chuyên biệt Scaler}:
\begin{itemize}
    \item \textbf{StandardScaler}: Tối ưu cho SVM và linear models với assumption dữ liệu có normal distribution - Zero-mean và unit variance normalization
    \item \textbf{MinMaxScaler}: Có lợi cho ensemble methods nhưng gây hại cho SVM - Scale data về range [0,1] nhưng sensitive với outliers
    \item \textbf{RobustScaler}: Hiệu suất ổn định cho hầu hết algorithms với resistance to outliers - Dùng median và quartiles thay vì mean/std để ít sensitive với extreme values
\end{itemize}

\subsection{Đề xuất Cấu hình Mô hình}\label{subsec:device-recommendations}

\subsubsection{Chiến lược Triển khai Production}

\textbf{Cấu hình Tối ưu Hiệu suất}:
\begin{itemize}
    \item \textbf{Mô hình Chính}: Random Forest với StandardScaler (3.17s training, 100\% accuracy) - Balance tốt nhất giữa tốc độ và accuracy cho production use cases
    \item \textbf{Mô hình Dự phòng}: LightGBM với RobustScaler (6.34s training, 100\% accuracy) - Backup option với memory efficiency và outlier robustness  
    \item \textbf{Mô hình Inference Nhanh}: Decision Tree (0.034s training, 99.0\% accuracy) - Ultra-fast predictions cho real-time applications và resource-constrained environments
\end{itemize}

\textbf{Cấu hình Độ chính xác Tối đa}:
\begin{itemize}
    \item \textbf{Chiến lược Ensemble}: Stacking Ensemble với Logistic Regression meta-learner để đạt perfect 100\% accuracy thông qua sophisticated meta-learning
    \item \textbf{Training Pipeline}: CatBoost base learners với hierarchical optimization để maximize individual model performance trước khi ensemble chúng lại
    \item \textbf{Trade-off Hiệu suất}: Chấp nhận thời gian training lâu hơn để đạt được perfect accuracy - Suitable cho critical medical applications where accuracy là top priority over speed
\end{itemize}

\subsubsection{Ví dụ Triển khai Mô hình AIO Classifier}

\textbf{Triển khai Kiến trúc BaseModel}:

\begin{minted}{python}
class BaseModel:
    """Abstract base class cho tất cả ML models"""
    
    def __init__(self, **kwargs):
        self.model = None
        self.is_fitted = False
        self.training_history = []
        self.model_params = {}
        
    def fit(self, X, y):
        """Abstract method - phải implement trong subclass"""
        pass
        
    def predict(self, X):
        """Abstract method - phải implement trong subclass"""
        pass
        
    def score(self, X, y):
        """Calculate model score"""
        pass
        
    def validate(self, X, y):
        """Validate model performance"""
        pass
\end{minted}

\textbf{Chú thích}: Nền tảng BaseModel cung cấp giao diện thống nhất cho tất cả các mô hình học máy trong nền tảng. Các phương thức trừu tượng đảm bảo tính nhất quán giữa các triển khai mô hình khác nhau.

\textbf{Triển khai Hệ thống Đăng ký Mô hình}:

\begin{minted}{python}
# models/register_models.py
def register_all_models(registry):
    """Register tất cả available models trong registry"""
    
    # Clustering models
    registry.register_model('kmeans', KMeansModel, {...})
    
    # Classification models  
    registry.register_model('knn', KNNModel, {...})
    registry.register_model('decision_tree', DecisionTreeModel, {...})
    registry.register_model('naive_bayes', NaiveBayesModel, {...})
    registry.register_model('svm', SVMModel, {...})
    registry.register_model('logistic_regression', LogisticRegressionModel, {...})
    registry.register_model('linear_svc', LinearSVCModel, {...})
    registry.register_model('random_forest', RandomForestModel, {...})
    registry.register_model('adaboost', AdaBoostModel, {...})
    registry.register_model('gradient_boosting', GradientBoostingModel, {...})
    registry.register_model('xgboost', XGBoostModel, {...})
    registry.register_model('lightgbm', LightGBMModel, {...})
    registry.register_model('catboost', CatBoostModel, {...})
    
    # Ensemble models
    registry.register_model('voting_ensemble_hard', EnsembleStackingClassifier, {...})
    registry.register_model('voting_ensemble_soft', EnsembleStackingClassifier, {...})
    registry.register_model('stacking_ensemble_logistic_regression', EnsembleStackingClassifier, {...})
\end{minted}

\textbf{Chú thích}: Hệ thống đăng ký mô hình cho phép đăng ký và quản lý mô hình động. Hệ thống hỗ trợ hơn 13 mô hình bao gồm clustering, classification, và các phương pháp ensemble với cấu hình linh hoạt.

\textbf{Triển khai Random Forest Nâng cao}:

\begin{minted}{python}
class RandomForestModel(BaseModel):
    """Random Forest với GPU-first configuration"""
    
    def __init__(self, **kwargs):
        super().__init__(**kwargs)
        
        # Default parameters
        default_params = {
            'n_estimators': 100,
            'max_depth': None,
            'max_features': 'sqrt',
            'min_samples_split': 2,
            'min_samples_leaf': 1,
            'bootstrap': True,
            'random_state': 42,
            'n_jobs': -1,  # Use all CPU cores
            'verbose': 0
        }
        
        default_params.update(kwargs)
        self.model_params = default_params
        
    def fit(self, X: Union[np.ndarray, sparse.csr_matrix], y: np.ndarray):
        """Fit Random Forest với multithreading"""
        
        self.model = RandomForestClassifier(**self.model_params)
        
        # Display multithreading info
        n_jobs = self.model_params.get('n_jobs', -1)
        if n_jobs == -1:
            import os
            cpu_count = os.cpu_count()
            print(f"🔄 CPU multithreading: Using all {cpu_count} available cores")
        else:
            print(f"🔄 CPU multithreading: Using {n_jobs} parallel jobs")
            
        self.model.fit(X, y)
        return self
        
    def get_feature_importance(self) -> np.ndarray:
        """Get feature importance từ Random Forest"""
        if hasattr(self.model, 'feature_importances_'):
            return self.model.feature_importances_
        return None
\end{minted}

\textbf{Chú thích}: RandomForestModel triển khai cấu hình tiên tiến với đa luồng CPU và trích xuất tầm quan trọng đặc trưng. Đây là một trong những mô hình hiệu suất cao nhất trong nền tảng với độ chính xác 100\% trên các tập dữ liệu tim mạch.

\textbf{Triển khai Tăng tốc GPU cho XGBoost}:

\begin{minted}{python}
class XGBoostModel(BaseModel):
    """XGBoost với GPU-first configuration"""
    
    def __init__(self, **kwargs):
        super().__init__(**kwargs)
        
        # Import XGBoost
        try:
            import xgboost as xgb
            self.xgb = xgb
        except ImportError:
            raise ImportError("XGBoost is required but not installed")
            
        # Default parameters
        default_params = {
            'n_estimators': 100,
            'max_depth': 6,
            'eta': 0.3,
            'subsample': 1.0,
            'colsample_bytree': 1.0,
            'min_child_weight': 1,
            'reg_lambda': 1.0,
            'reg_alpha': 0.0,
            'random_state': 42,
            'verbosity': 0
        }
        
        # Configure GPU/CPU based on device policy
        self._configure_device_params(default_params)
        
        default_params.update(kwargs)
        self.model_params = default_params
        
    def _configure_device_params(self, params: Dict[str, Any]):
        """Configure device-specific parameters"""
        try:
            from gpu_config_manager import configure_model_device
            
            device_config = configure_model_device("xgboost")
            
            if device_config["use_gpu"]:
                params.update(device_config["device_params"])
                print(f"🚀 XGBoost configured for GPU: {device_config['gpu_info']}")
            else:
                params.update({
                    "tree_method": "hist",
                    "predictor": "auto"
                })
                print(f"💻 XGBoost configured for CPU")
                
        except ImportError:
            # Fallback to CPU
            params.update({
                "tree_method": "hist",
                "predictor": "auto"
            })
            print(f"💻 XGBoost configured for CPU (fallback)")
\end{minted}

\textbf{Chú thích}: XGBoostModel triển khai gia tốc GPU tiên tiến với tự động chuyển đổi về CPU. Hệ thống tự động cấu hình các thiết lập tối ưu dựa trên tài nguyên phần cứng có sẵn.

\subsection{Tổng kết Model Improvements}

\noindent
Phân tích AIO Classifier từ 43 cấu hình mô hình cho thấy các hệ thống phân cấp hiệu suất rõ ràng với những cơ hội tối ưu hóa cụ thể. Các thuật toán dựa trên cây quyết định chiếm ưu thế trong các tác vụ dự đoán tim mạch, trong khi các phương pháp ensemble đạt được sự hoàn hảo thông qua học tập siêu mô hình tinh vi.

Những phát hiện này cung cấp những hiểu biết có thể áp dụng cho triển khai sản xuất, cân bằng các yêu cầu về thành công với hiệu quả tính toán. Các chiến lược caching và tối ưu hóa tiên tiến cho phép hỗ trợ nhiều cấu hình mô hình một cách hiệu quả trong các ứng dụng thực tế.