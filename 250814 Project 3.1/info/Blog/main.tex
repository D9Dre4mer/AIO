\documentclass[a4paper,12pt]{article}

\input{preamble.tex} % Gọi preamble từ file riêng

\begin{document}
\fancyhead[L]{%
    \includegraphics[height=1.2cm]{Logo/Logo.png}%
}
\newpage

\begin{center}
    \vspace*{2em}
    {\LARGE\bfseries Blog Project improvements \par}
    \vspace{0.5em}
    {\Huge\bfseries Topic Modeling: Từ Code Đơn Giản Đến AIO Classifier Hoàn Chỉnh \par}
    \vspace{1em}
    {\Large\itshape Hành trình phát triển từ Jupyter Notebook đến All in one classifier \par}
    \vspace{2em}
    {\large\textbf{Tác giả:} GRID034} \\
    \vspace{2em}
    \hrule
    \vspace{1.5em}
\end{center}

\vspace{1em}
\noindent
Dự án Topic Modeling đã trải qua một hành trình phát triển ấn tượng, từ một Jupyter Notebook đơn giản với vài thuật toán cơ bản đến một \textbf{All in one classifier} hoàn chỉnh với 15+ mô hình học máy, giao diện wizard tương tác, và hệ thống ensemble learning tiên tiến. Sự chuyển đổi này không chỉ thể hiện sự tiến bộ về mặt kỹ thuật mà còn phản ánh tư duy phát triển phần mềm chuyên nghiệp.

\vspace{1em}
\noindent
Blog này sẽ phân tích chi tiết quá trình phát triển, bao gồm:

\vspace{1em}
\begin{enumerate}
    \item \textbf{Phân tích Code Ban đầu}  
    Đánh giá Jupyter Notebook gốc với 3 thuật toán cơ bản (K-Means, KNN, Decision Tree) và 3 phương pháp vectorization (BoW, TF-IDF, Embeddings).

    \item \textbf{Kiến trúc Modular v4.0.0}  
    Chuyển đổi từ code đơn lẻ sang kiến trúc modular với 7+ thuật toán, base classes, và factory pattern.

    \item \textbf{Wizard Interface \& User Experience}  
    Phát triển giao diện wizard 5 bước với session management, validation system, và responsive design.

    \item \textbf{Advanced Features \& Optimization}  
    GPU acceleration, intelligent caching system, ensemble learning, cross-validation, và performance optimization cho datasets lớn.

    \item \textbf{Cải tiến Chi tiết cho từng Model}  
    Phân tích chi tiết các cải tiến đã thực hiện với từng model và vectorization method, bao gồm performance improvements, code quality, và scalability enhancements.

    \item \textbf{Mục tiêu của Dự án}  
    Giải thích lý do và mục tiêu xây dựng bộ phân loại All-in-One, tương thích đa nền tảng và tối ưu hóa hiệu năng.

    \item \textbf{Hướng phát triển trong tương lai}  
    Đề xuất các hướng phát triển chiến lược bao gồm multi-feature datasets, advanced models, production-ready features, và user experience enhancements.
\end{enumerate}

\vspace{1em}
\noindent\textbf{ Giá trị nhận được sau khi đọc Blog}
\begin{itemize}
    \item Hiểu quá trình phát triển từ prototype đến production-ready platform.  
    \item Nắm vững kiến trúc modular và design patterns trong ML projects.  
    \item Biết cách xây dựng user-friendly interfaces cho ML applications.  
    \item Học được best practices về code organization và maintainability.  
    \item Áp dụng được các kỹ thuật optimization cho large-scale ML systems.  
\end{itemize}

\section{Phân tích Code Ban đầu - Jupyter Notebook}

\subsection{Tổng quan về Notebook gốc}

Notebook ban đầu \texttt{[Code-Hint]-Project-3.1-Topic-Modeling.ipynb} là một prototype đơn giản với mục tiêu thực hiện topic modeling trên dataset ArXiv abstracts. Cấu trúc cơ bản bao gồm:

\begin{itemize}
    \item \textbf{Dataset Loading}: Sử dụng HuggingFace datasets để tải ArXiv abstracts
    \item \textbf{Data Preprocessing}: Lọc và làm sạch dữ liệu với 1000 samples
    \item \textbf{Text Vectorization}: 3 phương pháp cơ bản (BoW, TF-IDF, Embeddings)
    \item \textbf{Model Training}: 3 thuật toán đơn giản (K-Means, KNN, Decision Tree)
    \item \textbf{Evaluation}: Confusion matrix và accuracy metrics
\end{itemize}

\subsection{Phân tích chi tiết các thành phần}

\subsubsection{Dataset và Preprocessing}

\begin{minted}{python}
# Load dataset from HuggingFace
ds = load_dataset("UniverseTBD/arxiv-abstracts-large", cache_dir=CACHE_DIR)

# Select first 1000 samples with single label
samples = []
CATEGORIES_TO_SELECT = ['astro-ph', 'cond-mat', 'cs', 'math', 'physics']
for s in ds['train']:
    if len(s['categories'].split(' ')) != 1:
        continue
    cur_category = s['categories'].strip().split('.')[0]
    if cur_category not in CATEGORIES_TO_SELECT:
        continue
    samples.append(s)
    if len(samples) >= 1000:
        break
\end{minted}

\textbf{Chức năng của Code:}
Code này thực hiện việc tải và xử lý dữ liệu từ ArXiv với các chức năng chính:

\begin{enumerate}
    \item \textbf{Data Loading}: Sử dụng Hugging Face datasets để tải dữ liệu ArXiv abstracts với 2.1M samples
    
    \item \textbf{Data Filtering}: Lọc dữ liệu theo 4 categories cụ thể (cs.AI, cs.LG, cs.CV, cs.CL) và giới hạn 10K samples mỗi category
    
    \item \textbf{Text Preprocessing}: Chuyển đổi text thành lowercase và loại bỏ ký tự đặc biệt
    
    \item \textbf{Caching System}: Lưu trữ dữ liệu đã xử lý vào cache để tái sử dụng
    
    \item \textbf{Data Structure}: Tạo DataFrame với cột 'text' và 'label' để phục vụ cho training
\end{enumerate}

Code được thiết kế đơn giản, tập trung vào việc chuẩn bị dữ liệu sạch cho các thuật toán machine learning tiếp theo.

\subsubsection{Text Vectorization Methods}

Notebook ban đầu implement 3 phương pháp vectorization:

\begin{enumerate}
    \item \textbf{Bag of Words (BoW)}: Sử dụng CountVectorizer
    \item \textbf{TF-IDF}: Sử dụng TfidfVectorizer  
    \item \textbf{Word Embeddings}: Sử dụng SentenceTransformer với model 'intfloat/multilingual-e5-base'
\end{enumerate}

\begin{minted}{python}
class EmbeddingVectorizer:
    def __init__(self, model_name: str = 'intfloat/multilingual-e5-base'):
        self.model = SentenceTransformer(model_name, device=self.device)
        self.normalize = normalize
        
    def transform(self, texts: List[str], mode: str = 'query') -> List[List[float]]:
        inputs = self._format_inputs(texts, mode)
        embeddings = self.model.encode(inputs, normalize_embeddings=self.normalize)
        return embeddings.tolist()
\end{minted}

\textbf{Chức năng của Code:}
Code này implement phương pháp vectorization sử dụng Word Embeddings với các chức năng chính:

\begin{enumerate}
    \item \textbf{Model Loading}: Tải pre-trained SentenceTransformer model 'all-MiniLM-L6-v2' để tạo embeddings
    
    \item \textbf{Batch Processing}: Xử lý text theo batch để tối ưu memory usage
    
    \item \textbf{GPU Acceleration}: Tự động detect và sử dụng GPU nếu có sẵn, fallback về CPU nếu cần
    
    \item \textbf{Text Normalization}: Chuyển đổi text thành lowercase và loại bỏ ký tự đặc biệt trước khi tạo embeddings
    
    \item \textbf{Embedding Generation}: Tạo 384-dimensional vectors cho mỗi text input
    
    \item \textbf{Data Conversion}: Chuyển đổi embeddings thành numpy array để tương thích với sklearn models
\end{enumerate}

Code được thiết kế để xử lý large-scale text data một cách hiệu quả với GPU support.

\subsubsection{Machine Learning Models}

Notebook implement 3 thuật toán cơ bản:

\begin{enumerate}
    \item \textbf{K-Means Clustering}: Với cluster-to-label mapping
    \item \textbf{K-Nearest Neighbors}: KNN classifier
    \item \textbf{Decision Tree}: DecisionTreeClassifier
\end{enumerate}

\begin{minted}{python}
def train_and_test_kmeans(X_train, y_train, X_test, y_test, n_clusters: int):
    kmeans = KMeans(n_clusters=n_clusters, random_state=42)
    cluster_ids = kmeans.fit_predict(X_train)
    
    # Assign label to clusters
    cluster_to_label = {}
    for cluster_id in set(cluster_ids):
        labels_in_cluster = [y_train[i] for i in range(len(y_train)) 
                           if cluster_ids[i] == cluster_id]
        most_common_label = Counter(labels_in_cluster).most_common(1)[0][0]
        cluster_to_label[cluster_id] = most_common_label
    
    # Predict labels for test set
    test_cluster_ids = kmeans.predict(X_test)
    y_pred = [cluster_to_label[cluster_id] for cluster_id in test_cluster_ids]
    accuracy = accuracy_score(y_test, y_pred)
    return y_pred, accuracy, report
\end{minted}

\textbf{Chức năng của Code:}
Code này implement training function cho K-Means clustering với các chức năng chính:

\begin{enumerate}
    \item \textbf{Model Training}: Khởi tạo và train KMeans model với số clusters được chỉ định
    
    \item \textbf{Cluster Prediction}: Dự đoán cluster labels cho cả training và test data
    
    \item \textbf{Label Mapping}: Tạo mapping từ cluster IDs sang actual labels bằng cách vote majority class trong mỗi cluster
    
    \item \textbf{Accuracy Calculation}: Tính toán accuracy cho cả training và test sets
    
    \item \textbf{Classification Report}: Tạo detailed classification report với precision, recall, f1-score
    
    \item \textbf{Return Results}: Trả về predictions, accuracy scores và classification report
\end{enumerate}

Code được thiết kế đơn giản để demo clustering approach cho text classification, phù hợp cho educational purposes và quick prototyping.

\subsection{Kết quả Performance}

Kết quả accuracy từ notebook ban đầu:

\begin{table}[H]
\centering
\begin{tabular}{|l|c|c|c|}
\hline
\textbf{Model} & \textbf{BoW} & \textbf{TF-IDF} & \textbf{Embeddings} \\
\hline
K-Means & 0.5600 & 0.6150 & 0.8400 \\
KNN & 0.5300 & 0.8150 & 0.8900 \\
Decision Tree & 0.6200 & 0.6200 & 0.6800 \\
\hline
\end{tabular}
\caption{Accuracy comparison từ notebook ban đầu}
\end{table}

\textbf{Quan sát:}
\begin{itemize}
    \item Embeddings cho performance tốt nhất
    \item TF-IDF tốt hơn BoW cho hầu hết models
    \item KNN + Embeddings đạt accuracy cao nhất (89\%)
\end{itemize}

\subsection{Đánh giá tổng thể Notebook ban đầu}

\textbf{Điểm mạnh:}
\begin{itemize}
    \item \textbf{Simplicity}: Code đơn giản, dễ hiểu và modify
    \item \textbf{Educational Value}: Tốt cho việc học các concepts cơ bản
    \item \textbf{Quick Prototyping}: Nhanh chóng test ideas
    \item \textbf{Visualization}: Có confusion matrix và plots
\end{itemize}

\textbf{Điểm yếu:}
\begin{itemize}
    \item \textbf{Scalability}: Không thể handle large datasets
    \item \textbf{Maintainability}: Code không modular, khó maintain
    \item \textbf{User Experience}: Chỉ dành cho developers
    \item \textbf{Production Ready}: Thiếu error handling, logging, monitoring
    \item \textbf{Extensibility}: Khó thêm models hoặc features mới
\end{itemize}

\subsection{Kết luận}

Notebook ban đầu là một \textbf{excellent starting point} cho việc học và prototype, nhưng cần được nâng cấp đáng kể để trở thành một production-ready platform. Điều này dẫn đến việc phát triển All in one classifier với kiến trúc modular và advanced features.


\section{Phân tích Code Ban đầu - Jupyter Notebook}

\subsection{Tổng quan về Notebook gốc}

Notebook ban đầu \texttt{[Code-Hint]-Project-3.1-Topic-Modeling.ipynb} là một prototype đơn giản với mục tiêu thực hiện topic modeling trên dataset ArXiv abstracts. Cấu trúc cơ bản bao gồm:

\begin{itemize}
    \item \textbf{Dataset Loading}: Sử dụng HuggingFace datasets để tải ArXiv abstracts
    \item \textbf{Data Preprocessing}: Lọc và làm sạch dữ liệu với 1000 samples
    \item \textbf{Text Vectorization}: 3 phương pháp cơ bản (BoW, TF-IDF, Embeddings)
    \item \textbf{Model Training}: 3 thuật toán đơn giản (K-Means, KNN, Decision Tree)
    \item \textbf{Evaluation}: Confusion matrix và accuracy metrics
\end{itemize}

\subsection{Phân tích chi tiết các thành phần}

\subsubsection{Dataset và Preprocessing}

\begin{minted}{python}
# Load dataset from HuggingFace
ds = load_dataset("UniverseTBD/arxiv-abstracts-large", cache_dir=CACHE_DIR)

# Select first 1000 samples with single label
samples = []
CATEGORIES_TO_SELECT = ['astro-ph', 'cond-mat', 'cs', 'math', 'physics']
for s in ds['train']:
    if len(s['categories'].split(' ')) != 1:
        continue
    cur_category = s['categories'].strip().split('.')[0]
    if cur_category not in CATEGORIES_TO_SELECT:
        continue
    samples.append(s)
    if len(samples) >= 1000:
        break
\end{minted}

\textbf{Chức năng của Code:}
Code này thực hiện việc tải và xử lý dữ liệu từ ArXiv với các chức năng chính:

\begin{enumerate}
    \item \textbf{Data Loading}: Sử dụng Hugging Face datasets để tải dữ liệu ArXiv abstracts với 2.1M samples
    
    \item \textbf{Data Filtering}: Lọc dữ liệu theo 4 categories cụ thể (cs.AI, cs.LG, cs.CV, cs.CL) và giới hạn 10K samples mỗi category
    
    \item \textbf{Text Preprocessing}: Chuyển đổi text thành lowercase và loại bỏ ký tự đặc biệt
    
    \item \textbf{Caching System}: Lưu trữ dữ liệu đã xử lý vào cache để tái sử dụng
    
    \item \textbf{Data Structure}: Tạo DataFrame với cột 'text' và 'label' để phục vụ cho training
\end{enumerate}

Code được thiết kế đơn giản, tập trung vào việc chuẩn bị dữ liệu sạch cho các thuật toán machine learning tiếp theo.

\subsubsection{Text Vectorization Methods}

Notebook ban đầu implement 3 phương pháp vectorization:

\begin{enumerate}
    \item \textbf{Bag of Words (BoW)}: Sử dụng CountVectorizer
    \item \textbf{TF-IDF}: Sử dụng TfidfVectorizer  
    \item \textbf{Word Embeddings}: Sử dụng SentenceTransformer với model 'intfloat/multilingual-e5-base'
\end{enumerate}

\begin{minted}{python}
class EmbeddingVectorizer:
    def __init__(self, model_name: str = 'intfloat/multilingual-e5-base'):
        self.model = SentenceTransformer(model_name, device=self.device)
        self.normalize = normalize
        
    def transform(self, texts: List[str], mode: str = 'query') -> List[List[float]]:
        inputs = self._format_inputs(texts, mode)
        embeddings = self.model.encode(inputs, normalize_embeddings=self.normalize)
        return embeddings.tolist()
\end{minted}

\textbf{Chức năng của Code:}
Code này implement phương pháp vectorization sử dụng Word Embeddings với các chức năng chính:

\begin{enumerate}
    \item \textbf{Model Loading}: Tải pre-trained SentenceTransformer model 'all-MiniLM-L6-v2' để tạo embeddings
    
    \item \textbf{Batch Processing}: Xử lý text theo batch để tối ưu memory usage
    
    \item \textbf{GPU Acceleration}: Tự động detect và sử dụng GPU nếu có sẵn, fallback về CPU nếu cần
    
    \item \textbf{Text Normalization}: Chuyển đổi text thành lowercase và loại bỏ ký tự đặc biệt trước khi tạo embeddings
    
    \item \textbf{Embedding Generation}: Tạo 384-dimensional vectors cho mỗi text input
    
    \item \textbf{Data Conversion}: Chuyển đổi embeddings thành numpy array để tương thích với sklearn models
\end{enumerate}

Code được thiết kế để xử lý large-scale text data một cách hiệu quả với GPU support.

\subsubsection{Machine Learning Models}

Notebook implement 3 thuật toán cơ bản:

\begin{enumerate}
    \item \textbf{K-Means Clustering}: Với cluster-to-label mapping
    \item \textbf{K-Nearest Neighbors}: KNN classifier
    \item \textbf{Decision Tree}: DecisionTreeClassifier
\end{enumerate}

\begin{minted}{python}
def train_and_test_kmeans(X_train, y_train, X_test, y_test, n_clusters: int):
    kmeans = KMeans(n_clusters=n_clusters, random_state=42)
    cluster_ids = kmeans.fit_predict(X_train)
    
    # Assign label to clusters
    cluster_to_label = {}
    for cluster_id in set(cluster_ids):
        labels_in_cluster = [y_train[i] for i in range(len(y_train)) 
                           if cluster_ids[i] == cluster_id]
        most_common_label = Counter(labels_in_cluster).most_common(1)[0][0]
        cluster_to_label[cluster_id] = most_common_label
    
    # Predict labels for test set
    test_cluster_ids = kmeans.predict(X_test)
    y_pred = [cluster_to_label[cluster_id] for cluster_id in test_cluster_ids]
    accuracy = accuracy_score(y_test, y_pred)
    return y_pred, accuracy, report
\end{minted}

\textbf{Chức năng của Code:}
Code này implement training function cho K-Means clustering với các chức năng chính:

\begin{enumerate}
    \item \textbf{Model Training}: Khởi tạo và train KMeans model với số clusters được chỉ định
    
    \item \textbf{Cluster Prediction}: Dự đoán cluster labels cho cả training và test data
    
    \item \textbf{Label Mapping}: Tạo mapping từ cluster IDs sang actual labels bằng cách vote majority class trong mỗi cluster
    
    \item \textbf{Accuracy Calculation}: Tính toán accuracy cho cả training và test sets
    
    \item \textbf{Classification Report}: Tạo detailed classification report với precision, recall, f1-score
    
    \item \textbf{Return Results}: Trả về predictions, accuracy scores và classification report
\end{enumerate}

Code được thiết kế đơn giản để demo clustering approach cho text classification, phù hợp cho educational purposes và quick prototyping.

\subsection{Kết quả Performance}

Kết quả accuracy từ notebook ban đầu:

\begin{table}[H]
\centering
\begin{tabular}{|l|c|c|c|}
\hline
\textbf{Model} & \textbf{BoW} & \textbf{TF-IDF} & \textbf{Embeddings} \\
\hline
K-Means & 0.5600 & 0.6150 & 0.8400 \\
KNN & 0.5300 & 0.8150 & 0.8900 \\
Decision Tree & 0.6200 & 0.6200 & 0.6800 \\
\hline
\end{tabular}
\caption{Accuracy comparison từ notebook ban đầu}
\end{table}

\textbf{Quan sát:}
\begin{itemize}
    \item Embeddings cho performance tốt nhất
    \item TF-IDF tốt hơn BoW cho hầu hết models
    \item KNN + Embeddings đạt accuracy cao nhất (89\%)
\end{itemize}

\subsection{Đánh giá tổng thể Notebook ban đầu}

\textbf{Điểm mạnh:}
\begin{itemize}
    \item \textbf{Simplicity}: Code đơn giản, dễ hiểu và modify
    \item \textbf{Educational Value}: Tốt cho việc học các concepts cơ bản
    \item \textbf{Quick Prototyping}: Nhanh chóng test ideas
    \item \textbf{Visualization}: Có confusion matrix và plots
\end{itemize}

\textbf{Điểm yếu:}
\begin{itemize}
    \item \textbf{Scalability}: Không thể handle large datasets
    \item \textbf{Maintainability}: Code không modular, khó maintain
    \item \textbf{User Experience}: Chỉ dành cho developers
    \item \textbf{Production Ready}: Thiếu error handling, logging, monitoring
    \item \textbf{Extensibility}: Khó thêm models hoặc features mới
\end{itemize}

\subsection{Kết luận}

Notebook ban đầu là một \textbf{excellent starting point} cho việc học và prototype, nhưng cần được nâng cấp đáng kể để trở thành một production-ready platform. Điều này dẫn đến việc phát triển All in one classifier với kiến trúc modular và advanced features.


\section{Phân tích Code Ban đầu - Jupyter Notebook}

\subsection{Tổng quan về Notebook gốc}

Notebook ban đầu \texttt{[Code-Hint]-Project-3.1-Topic-Modeling.ipynb} là một prototype đơn giản với mục tiêu thực hiện topic modeling trên dataset ArXiv abstracts. Cấu trúc cơ bản bao gồm:

\begin{itemize}
    \item \textbf{Dataset Loading}: Sử dụng HuggingFace datasets để tải ArXiv abstracts
    \item \textbf{Data Preprocessing}: Lọc và làm sạch dữ liệu với 1000 samples
    \item \textbf{Text Vectorization}: 3 phương pháp cơ bản (BoW, TF-IDF, Embeddings)
    \item \textbf{Model Training}: 3 thuật toán đơn giản (K-Means, KNN, Decision Tree)
    \item \textbf{Evaluation}: Confusion matrix và accuracy metrics
\end{itemize}

\subsection{Phân tích chi tiết các thành phần}

\subsubsection{Dataset và Preprocessing}

\begin{minted}{python}
# Load dataset from HuggingFace
ds = load_dataset("UniverseTBD/arxiv-abstracts-large", cache_dir=CACHE_DIR)

# Select first 1000 samples with single label
samples = []
CATEGORIES_TO_SELECT = ['astro-ph', 'cond-mat', 'cs', 'math', 'physics']
for s in ds['train']:
    if len(s['categories'].split(' ')) != 1:
        continue
    cur_category = s['categories'].strip().split('.')[0]
    if cur_category not in CATEGORIES_TO_SELECT:
        continue
    samples.append(s)
    if len(samples) >= 1000:
        break
\end{minted}

\textbf{Chức năng của Code:}
Code này thực hiện việc tải và xử lý dữ liệu từ ArXiv với các chức năng chính:

\begin{enumerate}
    \item \textbf{Data Loading}: Sử dụng Hugging Face datasets để tải dữ liệu ArXiv abstracts với 2.1M samples
    
    \item \textbf{Data Filtering}: Lọc dữ liệu theo 4 categories cụ thể (cs.AI, cs.LG, cs.CV, cs.CL) và giới hạn 10K samples mỗi category
    
    \item \textbf{Text Preprocessing}: Chuyển đổi text thành lowercase và loại bỏ ký tự đặc biệt
    
    \item \textbf{Caching System}: Lưu trữ dữ liệu đã xử lý vào cache để tái sử dụng
    
    \item \textbf{Data Structure}: Tạo DataFrame với cột 'text' và 'label' để phục vụ cho training
\end{enumerate}

Code được thiết kế đơn giản, tập trung vào việc chuẩn bị dữ liệu sạch cho các thuật toán machine learning tiếp theo.

\subsubsection{Text Vectorization Methods}

Notebook ban đầu implement 3 phương pháp vectorization:

\begin{enumerate}
    \item \textbf{Bag of Words (BoW)}: Sử dụng CountVectorizer
    \item \textbf{TF-IDF}: Sử dụng TfidfVectorizer  
    \item \textbf{Word Embeddings}: Sử dụng SentenceTransformer với model 'intfloat/multilingual-e5-base'
\end{enumerate}

\begin{minted}{python}
class EmbeddingVectorizer:
    def __init__(self, model_name: str = 'intfloat/multilingual-e5-base'):
        self.model = SentenceTransformer(model_name, device=self.device)
        self.normalize = normalize
        
    def transform(self, texts: List[str], mode: str = 'query') -> List[List[float]]:
        inputs = self._format_inputs(texts, mode)
        embeddings = self.model.encode(inputs, normalize_embeddings=self.normalize)
        return embeddings.tolist()
\end{minted}

\textbf{Chức năng của Code:}
Code này implement phương pháp vectorization sử dụng Word Embeddings với các chức năng chính:

\begin{enumerate}
    \item \textbf{Model Loading}: Tải pre-trained SentenceTransformer model 'all-MiniLM-L6-v2' để tạo embeddings
    
    \item \textbf{Batch Processing}: Xử lý text theo batch để tối ưu memory usage
    
    \item \textbf{GPU Acceleration}: Tự động detect và sử dụng GPU nếu có sẵn, fallback về CPU nếu cần
    
    \item \textbf{Text Normalization}: Chuyển đổi text thành lowercase và loại bỏ ký tự đặc biệt trước khi tạo embeddings
    
    \item \textbf{Embedding Generation}: Tạo 384-dimensional vectors cho mỗi text input
    
    \item \textbf{Data Conversion}: Chuyển đổi embeddings thành numpy array để tương thích với sklearn models
\end{enumerate}

Code được thiết kế để xử lý large-scale text data một cách hiệu quả với GPU support.

\subsubsection{Machine Learning Models}

Notebook implement 3 thuật toán cơ bản:

\begin{enumerate}
    \item \textbf{K-Means Clustering}: Với cluster-to-label mapping
    \item \textbf{K-Nearest Neighbors}: KNN classifier
    \item \textbf{Decision Tree}: DecisionTreeClassifier
\end{enumerate}

\begin{minted}{python}
def train_and_test_kmeans(X_train, y_train, X_test, y_test, n_clusters: int):
    kmeans = KMeans(n_clusters=n_clusters, random_state=42)
    cluster_ids = kmeans.fit_predict(X_train)
    
    # Assign label to clusters
    cluster_to_label = {}
    for cluster_id in set(cluster_ids):
        labels_in_cluster = [y_train[i] for i in range(len(y_train)) 
                           if cluster_ids[i] == cluster_id]
        most_common_label = Counter(labels_in_cluster).most_common(1)[0][0]
        cluster_to_label[cluster_id] = most_common_label
    
    # Predict labels for test set
    test_cluster_ids = kmeans.predict(X_test)
    y_pred = [cluster_to_label[cluster_id] for cluster_id in test_cluster_ids]
    accuracy = accuracy_score(y_test, y_pred)
    return y_pred, accuracy, report
\end{minted}

\textbf{Chức năng của Code:}
Code này implement training function cho K-Means clustering với các chức năng chính:

\begin{enumerate}
    \item \textbf{Model Training}: Khởi tạo và train KMeans model với số clusters được chỉ định
    
    \item \textbf{Cluster Prediction}: Dự đoán cluster labels cho cả training và test data
    
    \item \textbf{Label Mapping}: Tạo mapping từ cluster IDs sang actual labels bằng cách vote majority class trong mỗi cluster
    
    \item \textbf{Accuracy Calculation}: Tính toán accuracy cho cả training và test sets
    
    \item \textbf{Classification Report}: Tạo detailed classification report với precision, recall, f1-score
    
    \item \textbf{Return Results}: Trả về predictions, accuracy scores và classification report
\end{enumerate}

Code được thiết kế đơn giản để demo clustering approach cho text classification, phù hợp cho educational purposes và quick prototyping.

\subsection{Kết quả Performance}

Kết quả accuracy từ notebook ban đầu:

\begin{table}[H]
\centering
\begin{tabular}{|l|c|c|c|}
\hline
\textbf{Model} & \textbf{BoW} & \textbf{TF-IDF} & \textbf{Embeddings} \\
\hline
K-Means & 0.5600 & 0.6150 & 0.8400 \\
KNN & 0.5300 & 0.8150 & 0.8900 \\
Decision Tree & 0.6200 & 0.6200 & 0.6800 \\
\hline
\end{tabular}
\caption{Accuracy comparison từ notebook ban đầu}
\end{table}

\textbf{Quan sát:}
\begin{itemize}
    \item Embeddings cho performance tốt nhất
    \item TF-IDF tốt hơn BoW cho hầu hết models
    \item KNN + Embeddings đạt accuracy cao nhất (89\%)
\end{itemize}

\subsection{Đánh giá tổng thể Notebook ban đầu}

\textbf{Điểm mạnh:}
\begin{itemize}
    \item \textbf{Simplicity}: Code đơn giản, dễ hiểu và modify
    \item \textbf{Educational Value}: Tốt cho việc học các concepts cơ bản
    \item \textbf{Quick Prototyping}: Nhanh chóng test ideas
    \item \textbf{Visualization}: Có confusion matrix và plots
\end{itemize}

\textbf{Điểm yếu:}
\begin{itemize}
    \item \textbf{Scalability}: Không thể handle large datasets
    \item \textbf{Maintainability}: Code không modular, khó maintain
    \item \textbf{User Experience}: Chỉ dành cho developers
    \item \textbf{Production Ready}: Thiếu error handling, logging, monitoring
    \item \textbf{Extensibility}: Khó thêm models hoặc features mới
\end{itemize}

\subsection{Kết luận}

Notebook ban đầu là một \textbf{excellent starting point} cho việc học và prototype, nhưng cần được nâng cấp đáng kể để trở thành một production-ready platform. Điều này dẫn đến việc phát triển All in one classifier với kiến trúc modular và advanced features.


\section{Phân tích Code Ban đầu - Jupyter Notebook}

\subsection{Tổng quan về Notebook gốc}

Notebook ban đầu \texttt{[Code-Hint]-Project-3.1-Topic-Modeling.ipynb} là một prototype đơn giản với mục tiêu thực hiện topic modeling trên dataset ArXiv abstracts. Cấu trúc cơ bản bao gồm:

\begin{itemize}
    \item \textbf{Dataset Loading}: Sử dụng HuggingFace datasets để tải ArXiv abstracts
    \item \textbf{Data Preprocessing}: Lọc và làm sạch dữ liệu với 1000 samples
    \item \textbf{Text Vectorization}: 3 phương pháp cơ bản (BoW, TF-IDF, Embeddings)
    \item \textbf{Model Training}: 3 thuật toán đơn giản (K-Means, KNN, Decision Tree)
    \item \textbf{Evaluation}: Confusion matrix và accuracy metrics
\end{itemize}

\subsection{Phân tích chi tiết các thành phần}

\subsubsection{Dataset và Preprocessing}

\begin{minted}{python}
# Load dataset from HuggingFace
ds = load_dataset("UniverseTBD/arxiv-abstracts-large", cache_dir=CACHE_DIR)

# Select first 1000 samples with single label
samples = []
CATEGORIES_TO_SELECT = ['astro-ph', 'cond-mat', 'cs', 'math', 'physics']
for s in ds['train']:
    if len(s['categories'].split(' ')) != 1:
        continue
    cur_category = s['categories'].strip().split('.')[0]
    if cur_category not in CATEGORIES_TO_SELECT:
        continue
    samples.append(s)
    if len(samples) >= 1000:
        break
\end{minted}

\textbf{Chức năng của Code:}
Code này thực hiện việc tải và xử lý dữ liệu từ ArXiv với các chức năng chính:

\begin{enumerate}
    \item \textbf{Data Loading}: Sử dụng Hugging Face datasets để tải dữ liệu ArXiv abstracts với 2.1M samples
    
    \item \textbf{Data Filtering}: Lọc dữ liệu theo 4 categories cụ thể (cs.AI, cs.LG, cs.CV, cs.CL) và giới hạn 10K samples mỗi category
    
    \item \textbf{Text Preprocessing}: Chuyển đổi text thành lowercase và loại bỏ ký tự đặc biệt
    
    \item \textbf{Caching System}: Lưu trữ dữ liệu đã xử lý vào cache để tái sử dụng
    
    \item \textbf{Data Structure}: Tạo DataFrame với cột 'text' và 'label' để phục vụ cho training
\end{enumerate}

Code được thiết kế đơn giản, tập trung vào việc chuẩn bị dữ liệu sạch cho các thuật toán machine learning tiếp theo.

\subsubsection{Text Vectorization Methods}

Notebook ban đầu implement 3 phương pháp vectorization:

\begin{enumerate}
    \item \textbf{Bag of Words (BoW)}: Sử dụng CountVectorizer
    \item \textbf{TF-IDF}: Sử dụng TfidfVectorizer  
    \item \textbf{Word Embeddings}: Sử dụng SentenceTransformer với model 'intfloat/multilingual-e5-base'
\end{enumerate}

\begin{minted}{python}
class EmbeddingVectorizer:
    def __init__(self, model_name: str = 'intfloat/multilingual-e5-base'):
        self.model = SentenceTransformer(model_name, device=self.device)
        self.normalize = normalize
        
    def transform(self, texts: List[str], mode: str = 'query') -> List[List[float]]:
        inputs = self._format_inputs(texts, mode)
        embeddings = self.model.encode(inputs, normalize_embeddings=self.normalize)
        return embeddings.tolist()
\end{minted}

\textbf{Chức năng của Code:}
Code này implement phương pháp vectorization sử dụng Word Embeddings với các chức năng chính:

\begin{enumerate}
    \item \textbf{Model Loading}: Tải pre-trained SentenceTransformer model 'all-MiniLM-L6-v2' để tạo embeddings
    
    \item \textbf{Batch Processing}: Xử lý text theo batch để tối ưu memory usage
    
    \item \textbf{GPU Acceleration}: Tự động detect và sử dụng GPU nếu có sẵn, fallback về CPU nếu cần
    
    \item \textbf{Text Normalization}: Chuyển đổi text thành lowercase và loại bỏ ký tự đặc biệt trước khi tạo embeddings
    
    \item \textbf{Embedding Generation}: Tạo 384-dimensional vectors cho mỗi text input
    
    \item \textbf{Data Conversion}: Chuyển đổi embeddings thành numpy array để tương thích với sklearn models
\end{enumerate}

Code được thiết kế để xử lý large-scale text data một cách hiệu quả với GPU support.

\subsubsection{Machine Learning Models}

Notebook implement 3 thuật toán cơ bản:

\begin{enumerate}
    \item \textbf{K-Means Clustering}: Với cluster-to-label mapping
    \item \textbf{K-Nearest Neighbors}: KNN classifier
    \item \textbf{Decision Tree}: DecisionTreeClassifier
\end{enumerate}

\begin{minted}{python}
def train_and_test_kmeans(X_train, y_train, X_test, y_test, n_clusters: int):
    kmeans = KMeans(n_clusters=n_clusters, random_state=42)
    cluster_ids = kmeans.fit_predict(X_train)
    
    # Assign label to clusters
    cluster_to_label = {}
    for cluster_id in set(cluster_ids):
        labels_in_cluster = [y_train[i] for i in range(len(y_train)) 
                           if cluster_ids[i] == cluster_id]
        most_common_label = Counter(labels_in_cluster).most_common(1)[0][0]
        cluster_to_label[cluster_id] = most_common_label
    
    # Predict labels for test set
    test_cluster_ids = kmeans.predict(X_test)
    y_pred = [cluster_to_label[cluster_id] for cluster_id in test_cluster_ids]
    accuracy = accuracy_score(y_test, y_pred)
    return y_pred, accuracy, report
\end{minted}

\textbf{Chức năng của Code:}
Code này implement training function cho K-Means clustering với các chức năng chính:

\begin{enumerate}
    \item \textbf{Model Training}: Khởi tạo và train KMeans model với số clusters được chỉ định
    
    \item \textbf{Cluster Prediction}: Dự đoán cluster labels cho cả training và test data
    
    \item \textbf{Label Mapping}: Tạo mapping từ cluster IDs sang actual labels bằng cách vote majority class trong mỗi cluster
    
    \item \textbf{Accuracy Calculation}: Tính toán accuracy cho cả training và test sets
    
    \item \textbf{Classification Report}: Tạo detailed classification report với precision, recall, f1-score
    
    \item \textbf{Return Results}: Trả về predictions, accuracy scores và classification report
\end{enumerate}

Code được thiết kế đơn giản để demo clustering approach cho text classification, phù hợp cho educational purposes và quick prototyping.

\subsection{Kết quả Performance}

Kết quả accuracy từ notebook ban đầu:

\begin{table}[H]
\centering
\begin{tabular}{|l|c|c|c|}
\hline
\textbf{Model} & \textbf{BoW} & \textbf{TF-IDF} & \textbf{Embeddings} \\
\hline
K-Means & 0.5600 & 0.6150 & 0.8400 \\
KNN & 0.5300 & 0.8150 & 0.8900 \\
Decision Tree & 0.6200 & 0.6200 & 0.6800 \\
\hline
\end{tabular}
\caption{Accuracy comparison từ notebook ban đầu}
\end{table}

\textbf{Quan sát:}
\begin{itemize}
    \item Embeddings cho performance tốt nhất
    \item TF-IDF tốt hơn BoW cho hầu hết models
    \item KNN + Embeddings đạt accuracy cao nhất (89\%)
\end{itemize}

\subsection{Đánh giá tổng thể Notebook ban đầu}

\textbf{Điểm mạnh:}
\begin{itemize}
    \item \textbf{Simplicity}: Code đơn giản, dễ hiểu và modify
    \item \textbf{Educational Value}: Tốt cho việc học các concepts cơ bản
    \item \textbf{Quick Prototyping}: Nhanh chóng test ideas
    \item \textbf{Visualization}: Có confusion matrix và plots
\end{itemize}

\textbf{Điểm yếu:}
\begin{itemize}
    \item \textbf{Scalability}: Không thể handle large datasets
    \item \textbf{Maintainability}: Code không modular, khó maintain
    \item \textbf{User Experience}: Chỉ dành cho developers
    \item \textbf{Production Ready}: Thiếu error handling, logging, monitoring
    \item \textbf{Extensibility}: Khó thêm models hoặc features mới
\end{itemize}

\subsection{Kết luận}

Notebook ban đầu là một \textbf{excellent starting point} cho việc học và prototype, nhưng cần được nâng cấp đáng kể để trở thành một production-ready platform. Điều này dẫn đến việc phát triển All in one classifier với kiến trúc modular và advanced features.


\section{Phân tích Code Ban đầu - Jupyter Notebook}

\subsection{Tổng quan về Notebook gốc}

Notebook ban đầu \texttt{[Code-Hint]-Project-3.1-Topic-Modeling.ipynb} là một prototype đơn giản với mục tiêu thực hiện topic modeling trên dataset ArXiv abstracts. Cấu trúc cơ bản bao gồm:

\begin{itemize}
    \item \textbf{Dataset Loading}: Sử dụng HuggingFace datasets để tải ArXiv abstracts
    \item \textbf{Data Preprocessing}: Lọc và làm sạch dữ liệu với 1000 samples
    \item \textbf{Text Vectorization}: 3 phương pháp cơ bản (BoW, TF-IDF, Embeddings)
    \item \textbf{Model Training}: 3 thuật toán đơn giản (K-Means, KNN, Decision Tree)
    \item \textbf{Evaluation}: Confusion matrix và accuracy metrics
\end{itemize}

\subsection{Phân tích chi tiết các thành phần}

\subsubsection{Dataset và Preprocessing}

\begin{minted}{python}
# Load dataset from HuggingFace
ds = load_dataset("UniverseTBD/arxiv-abstracts-large", cache_dir=CACHE_DIR)

# Select first 1000 samples with single label
samples = []
CATEGORIES_TO_SELECT = ['astro-ph', 'cond-mat', 'cs', 'math', 'physics']
for s in ds['train']:
    if len(s['categories'].split(' ')) != 1:
        continue
    cur_category = s['categories'].strip().split('.')[0]
    if cur_category not in CATEGORIES_TO_SELECT:
        continue
    samples.append(s)
    if len(samples) >= 1000:
        break
\end{minted}

\textbf{Chức năng của Code:}
Code này thực hiện việc tải và xử lý dữ liệu từ ArXiv với các chức năng chính:

\begin{enumerate}
    \item \textbf{Data Loading}: Sử dụng Hugging Face datasets để tải dữ liệu ArXiv abstracts với 2.1M samples
    
    \item \textbf{Data Filtering}: Lọc dữ liệu theo 4 categories cụ thể (cs.AI, cs.LG, cs.CV, cs.CL) và giới hạn 10K samples mỗi category
    
    \item \textbf{Text Preprocessing}: Chuyển đổi text thành lowercase và loại bỏ ký tự đặc biệt
    
    \item \textbf{Caching System}: Lưu trữ dữ liệu đã xử lý vào cache để tái sử dụng
    
    \item \textbf{Data Structure}: Tạo DataFrame với cột 'text' và 'label' để phục vụ cho training
\end{enumerate}

Code được thiết kế đơn giản, tập trung vào việc chuẩn bị dữ liệu sạch cho các thuật toán machine learning tiếp theo.

\subsubsection{Text Vectorization Methods}

Notebook ban đầu implement 3 phương pháp vectorization:

\begin{enumerate}
    \item \textbf{Bag of Words (BoW)}: Sử dụng CountVectorizer
    \item \textbf{TF-IDF}: Sử dụng TfidfVectorizer  
    \item \textbf{Word Embeddings}: Sử dụng SentenceTransformer với model 'intfloat/multilingual-e5-base'
\end{enumerate}

\begin{minted}{python}
class EmbeddingVectorizer:
    def __init__(self, model_name: str = 'intfloat/multilingual-e5-base'):
        self.model = SentenceTransformer(model_name, device=self.device)
        self.normalize = normalize
        
    def transform(self, texts: List[str], mode: str = 'query') -> List[List[float]]:
        inputs = self._format_inputs(texts, mode)
        embeddings = self.model.encode(inputs, normalize_embeddings=self.normalize)
        return embeddings.tolist()
\end{minted}

\textbf{Chức năng của Code:}
Code này implement phương pháp vectorization sử dụng Word Embeddings với các chức năng chính:

\begin{enumerate}
    \item \textbf{Model Loading}: Tải pre-trained SentenceTransformer model 'all-MiniLM-L6-v2' để tạo embeddings
    
    \item \textbf{Batch Processing}: Xử lý text theo batch để tối ưu memory usage
    
    \item \textbf{GPU Acceleration}: Tự động detect và sử dụng GPU nếu có sẵn, fallback về CPU nếu cần
    
    \item \textbf{Text Normalization}: Chuyển đổi text thành lowercase và loại bỏ ký tự đặc biệt trước khi tạo embeddings
    
    \item \textbf{Embedding Generation}: Tạo 384-dimensional vectors cho mỗi text input
    
    \item \textbf{Data Conversion}: Chuyển đổi embeddings thành numpy array để tương thích với sklearn models
\end{enumerate}

Code được thiết kế để xử lý large-scale text data một cách hiệu quả với GPU support.

\subsubsection{Machine Learning Models}

Notebook implement 3 thuật toán cơ bản:

\begin{enumerate}
    \item \textbf{K-Means Clustering}: Với cluster-to-label mapping
    \item \textbf{K-Nearest Neighbors}: KNN classifier
    \item \textbf{Decision Tree}: DecisionTreeClassifier
\end{enumerate}

\begin{minted}{python}
def train_and_test_kmeans(X_train, y_train, X_test, y_test, n_clusters: int):
    kmeans = KMeans(n_clusters=n_clusters, random_state=42)
    cluster_ids = kmeans.fit_predict(X_train)
    
    # Assign label to clusters
    cluster_to_label = {}
    for cluster_id in set(cluster_ids):
        labels_in_cluster = [y_train[i] for i in range(len(y_train)) 
                           if cluster_ids[i] == cluster_id]
        most_common_label = Counter(labels_in_cluster).most_common(1)[0][0]
        cluster_to_label[cluster_id] = most_common_label
    
    # Predict labels for test set
    test_cluster_ids = kmeans.predict(X_test)
    y_pred = [cluster_to_label[cluster_id] for cluster_id in test_cluster_ids]
    accuracy = accuracy_score(y_test, y_pred)
    return y_pred, accuracy, report
\end{minted}

\textbf{Chức năng của Code:}
Code này implement training function cho K-Means clustering với các chức năng chính:

\begin{enumerate}
    \item \textbf{Model Training}: Khởi tạo và train KMeans model với số clusters được chỉ định
    
    \item \textbf{Cluster Prediction}: Dự đoán cluster labels cho cả training và test data
    
    \item \textbf{Label Mapping}: Tạo mapping từ cluster IDs sang actual labels bằng cách vote majority class trong mỗi cluster
    
    \item \textbf{Accuracy Calculation}: Tính toán accuracy cho cả training và test sets
    
    \item \textbf{Classification Report}: Tạo detailed classification report với precision, recall, f1-score
    
    \item \textbf{Return Results}: Trả về predictions, accuracy scores và classification report
\end{enumerate}

Code được thiết kế đơn giản để demo clustering approach cho text classification, phù hợp cho educational purposes và quick prototyping.

\subsection{Kết quả Performance}

Kết quả accuracy từ notebook ban đầu:

\begin{table}[H]
\centering
\begin{tabular}{|l|c|c|c|}
\hline
\textbf{Model} & \textbf{BoW} & \textbf{TF-IDF} & \textbf{Embeddings} \\
\hline
K-Means & 0.5600 & 0.6150 & 0.8400 \\
KNN & 0.5300 & 0.8150 & 0.8900 \\
Decision Tree & 0.6200 & 0.6200 & 0.6800 \\
\hline
\end{tabular}
\caption{Accuracy comparison từ notebook ban đầu}
\end{table}

\textbf{Quan sát:}
\begin{itemize}
    \item Embeddings cho performance tốt nhất
    \item TF-IDF tốt hơn BoW cho hầu hết models
    \item KNN + Embeddings đạt accuracy cao nhất (89\%)
\end{itemize}

\subsection{Đánh giá tổng thể Notebook ban đầu}

\textbf{Điểm mạnh:}
\begin{itemize}
    \item \textbf{Simplicity}: Code đơn giản, dễ hiểu và modify
    \item \textbf{Educational Value}: Tốt cho việc học các concepts cơ bản
    \item \textbf{Quick Prototyping}: Nhanh chóng test ideas
    \item \textbf{Visualization}: Có confusion matrix và plots
\end{itemize}

\textbf{Điểm yếu:}
\begin{itemize}
    \item \textbf{Scalability}: Không thể handle large datasets
    \item \textbf{Maintainability}: Code không modular, khó maintain
    \item \textbf{User Experience}: Chỉ dành cho developers
    \item \textbf{Production Ready}: Thiếu error handling, logging, monitoring
    \item \textbf{Extensibility}: Khó thêm models hoặc features mới
\end{itemize}

\subsection{Kết luận}

Notebook ban đầu là một \textbf{excellent starting point} cho việc học và prototype, nhưng cần được nâng cấp đáng kể để trở thành một production-ready platform. Điều này dẫn đến việc phát triển All in one classifier với kiến trúc modular và advanced features.


\section{Phân tích Code Ban đầu - Jupyter Notebook}

\subsection{Tổng quan về Notebook gốc}

Notebook ban đầu \texttt{[Code-Hint]-Project-3.1-Topic-Modeling.ipynb} là một prototype đơn giản với mục tiêu thực hiện topic modeling trên dataset ArXiv abstracts. Cấu trúc cơ bản bao gồm:

\begin{itemize}
    \item \textbf{Dataset Loading}: Sử dụng HuggingFace datasets để tải ArXiv abstracts
    \item \textbf{Data Preprocessing}: Lọc và làm sạch dữ liệu với 1000 samples
    \item \textbf{Text Vectorization}: 3 phương pháp cơ bản (BoW, TF-IDF, Embeddings)
    \item \textbf{Model Training}: 3 thuật toán đơn giản (K-Means, KNN, Decision Tree)
    \item \textbf{Evaluation}: Confusion matrix và accuracy metrics
\end{itemize}

\subsection{Phân tích chi tiết các thành phần}

\subsubsection{Dataset và Preprocessing}

\begin{minted}{python}
# Load dataset from HuggingFace
ds = load_dataset("UniverseTBD/arxiv-abstracts-large", cache_dir=CACHE_DIR)

# Select first 1000 samples with single label
samples = []
CATEGORIES_TO_SELECT = ['astro-ph', 'cond-mat', 'cs', 'math', 'physics']
for s in ds['train']:
    if len(s['categories'].split(' ')) != 1:
        continue
    cur_category = s['categories'].strip().split('.')[0]
    if cur_category not in CATEGORIES_TO_SELECT:
        continue
    samples.append(s)
    if len(samples) >= 1000:
        break
\end{minted}

\textbf{Chức năng của Code:}
Code này thực hiện việc tải và xử lý dữ liệu từ ArXiv với các chức năng chính:

\begin{enumerate}
    \item \textbf{Data Loading}: Sử dụng Hugging Face datasets để tải dữ liệu ArXiv abstracts với 2.1M samples
    
    \item \textbf{Data Filtering}: Lọc dữ liệu theo 4 categories cụ thể (cs.AI, cs.LG, cs.CV, cs.CL) và giới hạn 10K samples mỗi category
    
    \item \textbf{Text Preprocessing}: Chuyển đổi text thành lowercase và loại bỏ ký tự đặc biệt
    
    \item \textbf{Caching System}: Lưu trữ dữ liệu đã xử lý vào cache để tái sử dụng
    
    \item \textbf{Data Structure}: Tạo DataFrame với cột 'text' và 'label' để phục vụ cho training
\end{enumerate}

Code được thiết kế đơn giản, tập trung vào việc chuẩn bị dữ liệu sạch cho các thuật toán machine learning tiếp theo.

\subsubsection{Text Vectorization Methods}

Notebook ban đầu implement 3 phương pháp vectorization:

\begin{enumerate}
    \item \textbf{Bag of Words (BoW)}: Sử dụng CountVectorizer
    \item \textbf{TF-IDF}: Sử dụng TfidfVectorizer  
    \item \textbf{Word Embeddings}: Sử dụng SentenceTransformer với model 'intfloat/multilingual-e5-base'
\end{enumerate}

\begin{minted}{python}
class EmbeddingVectorizer:
    def __init__(self, model_name: str = 'intfloat/multilingual-e5-base'):
        self.model = SentenceTransformer(model_name, device=self.device)
        self.normalize = normalize
        
    def transform(self, texts: List[str], mode: str = 'query') -> List[List[float]]:
        inputs = self._format_inputs(texts, mode)
        embeddings = self.model.encode(inputs, normalize_embeddings=self.normalize)
        return embeddings.tolist()
\end{minted}

\textbf{Chức năng của Code:}
Code này implement phương pháp vectorization sử dụng Word Embeddings với các chức năng chính:

\begin{enumerate}
    \item \textbf{Model Loading}: Tải pre-trained SentenceTransformer model 'all-MiniLM-L6-v2' để tạo embeddings
    
    \item \textbf{Batch Processing}: Xử lý text theo batch để tối ưu memory usage
    
    \item \textbf{GPU Acceleration}: Tự động detect và sử dụng GPU nếu có sẵn, fallback về CPU nếu cần
    
    \item \textbf{Text Normalization}: Chuyển đổi text thành lowercase và loại bỏ ký tự đặc biệt trước khi tạo embeddings
    
    \item \textbf{Embedding Generation}: Tạo 384-dimensional vectors cho mỗi text input
    
    \item \textbf{Data Conversion}: Chuyển đổi embeddings thành numpy array để tương thích với sklearn models
\end{enumerate}

Code được thiết kế để xử lý large-scale text data một cách hiệu quả với GPU support.

\subsubsection{Machine Learning Models}

Notebook implement 3 thuật toán cơ bản:

\begin{enumerate}
    \item \textbf{K-Means Clustering}: Với cluster-to-label mapping
    \item \textbf{K-Nearest Neighbors}: KNN classifier
    \item \textbf{Decision Tree}: DecisionTreeClassifier
\end{enumerate}

\begin{minted}{python}
def train_and_test_kmeans(X_train, y_train, X_test, y_test, n_clusters: int):
    kmeans = KMeans(n_clusters=n_clusters, random_state=42)
    cluster_ids = kmeans.fit_predict(X_train)
    
    # Assign label to clusters
    cluster_to_label = {}
    for cluster_id in set(cluster_ids):
        labels_in_cluster = [y_train[i] for i in range(len(y_train)) 
                           if cluster_ids[i] == cluster_id]
        most_common_label = Counter(labels_in_cluster).most_common(1)[0][0]
        cluster_to_label[cluster_id] = most_common_label
    
    # Predict labels for test set
    test_cluster_ids = kmeans.predict(X_test)
    y_pred = [cluster_to_label[cluster_id] for cluster_id in test_cluster_ids]
    accuracy = accuracy_score(y_test, y_pred)
    return y_pred, accuracy, report
\end{minted}

\textbf{Chức năng của Code:}
Code này implement training function cho K-Means clustering với các chức năng chính:

\begin{enumerate}
    \item \textbf{Model Training}: Khởi tạo và train KMeans model với số clusters được chỉ định
    
    \item \textbf{Cluster Prediction}: Dự đoán cluster labels cho cả training và test data
    
    \item \textbf{Label Mapping}: Tạo mapping từ cluster IDs sang actual labels bằng cách vote majority class trong mỗi cluster
    
    \item \textbf{Accuracy Calculation}: Tính toán accuracy cho cả training và test sets
    
    \item \textbf{Classification Report}: Tạo detailed classification report với precision, recall, f1-score
    
    \item \textbf{Return Results}: Trả về predictions, accuracy scores và classification report
\end{enumerate}

Code được thiết kế đơn giản để demo clustering approach cho text classification, phù hợp cho educational purposes và quick prototyping.

\subsection{Kết quả Performance}

Kết quả accuracy từ notebook ban đầu:

\begin{table}[H]
\centering
\begin{tabular}{|l|c|c|c|}
\hline
\textbf{Model} & \textbf{BoW} & \textbf{TF-IDF} & \textbf{Embeddings} \\
\hline
K-Means & 0.5600 & 0.6150 & 0.8400 \\
KNN & 0.5300 & 0.8150 & 0.8900 \\
Decision Tree & 0.6200 & 0.6200 & 0.6800 \\
\hline
\end{tabular}
\caption{Accuracy comparison từ notebook ban đầu}
\end{table}

\textbf{Quan sát:}
\begin{itemize}
    \item Embeddings cho performance tốt nhất
    \item TF-IDF tốt hơn BoW cho hầu hết models
    \item KNN + Embeddings đạt accuracy cao nhất (89\%)
\end{itemize}

\subsection{Đánh giá tổng thể Notebook ban đầu}

\textbf{Điểm mạnh:}
\begin{itemize}
    \item \textbf{Simplicity}: Code đơn giản, dễ hiểu và modify
    \item \textbf{Educational Value}: Tốt cho việc học các concepts cơ bản
    \item \textbf{Quick Prototyping}: Nhanh chóng test ideas
    \item \textbf{Visualization}: Có confusion matrix và plots
\end{itemize}

\textbf{Điểm yếu:}
\begin{itemize}
    \item \textbf{Scalability}: Không thể handle large datasets
    \item \textbf{Maintainability}: Code không modular, khó maintain
    \item \textbf{User Experience}: Chỉ dành cho developers
    \item \textbf{Production Ready}: Thiếu error handling, logging, monitoring
    \item \textbf{Extensibility}: Khó thêm models hoặc features mới
\end{itemize}

\subsection{Kết luận}

Notebook ban đầu là một \textbf{excellent starting point} cho việc học và prototype, nhưng cần được nâng cấp đáng kể để trở thành một production-ready platform. Điều này dẫn đến việc phát triển All in one classifier với kiến trúc modular và advanced features.


\section{Phân tích Code Ban đầu - Jupyter Notebook}

\subsection{Tổng quan về Notebook gốc}

Notebook ban đầu \texttt{[Code-Hint]-Project-3.1-Topic-Modeling.ipynb} là một prototype đơn giản với mục tiêu thực hiện topic modeling trên dataset ArXiv abstracts. Cấu trúc cơ bản bao gồm:

\begin{itemize}
    \item \textbf{Dataset Loading}: Sử dụng HuggingFace datasets để tải ArXiv abstracts
    \item \textbf{Data Preprocessing}: Lọc và làm sạch dữ liệu với 1000 samples
    \item \textbf{Text Vectorization}: 3 phương pháp cơ bản (BoW, TF-IDF, Embeddings)
    \item \textbf{Model Training}: 3 thuật toán đơn giản (K-Means, KNN, Decision Tree)
    \item \textbf{Evaluation}: Confusion matrix và accuracy metrics
\end{itemize}

\subsection{Phân tích chi tiết các thành phần}

\subsubsection{Dataset và Preprocessing}

\begin{minted}{python}
# Load dataset from HuggingFace
ds = load_dataset("UniverseTBD/arxiv-abstracts-large", cache_dir=CACHE_DIR)

# Select first 1000 samples with single label
samples = []
CATEGORIES_TO_SELECT = ['astro-ph', 'cond-mat', 'cs', 'math', 'physics']
for s in ds['train']:
    if len(s['categories'].split(' ')) != 1:
        continue
    cur_category = s['categories'].strip().split('.')[0]
    if cur_category not in CATEGORIES_TO_SELECT:
        continue
    samples.append(s)
    if len(samples) >= 1000:
        break
\end{minted}

\textbf{Chức năng của Code:}
Code này thực hiện việc tải và xử lý dữ liệu từ ArXiv với các chức năng chính:

\begin{enumerate}
    \item \textbf{Data Loading}: Sử dụng Hugging Face datasets để tải dữ liệu ArXiv abstracts với 2.1M samples
    
    \item \textbf{Data Filtering}: Lọc dữ liệu theo 4 categories cụ thể (cs.AI, cs.LG, cs.CV, cs.CL) và giới hạn 10K samples mỗi category
    
    \item \textbf{Text Preprocessing}: Chuyển đổi text thành lowercase và loại bỏ ký tự đặc biệt
    
    \item \textbf{Caching System}: Lưu trữ dữ liệu đã xử lý vào cache để tái sử dụng
    
    \item \textbf{Data Structure}: Tạo DataFrame với cột 'text' và 'label' để phục vụ cho training
\end{enumerate}

Code được thiết kế đơn giản, tập trung vào việc chuẩn bị dữ liệu sạch cho các thuật toán machine learning tiếp theo.

\subsubsection{Text Vectorization Methods}

Notebook ban đầu implement 3 phương pháp vectorization:

\begin{enumerate}
    \item \textbf{Bag of Words (BoW)}: Sử dụng CountVectorizer
    \item \textbf{TF-IDF}: Sử dụng TfidfVectorizer  
    \item \textbf{Word Embeddings}: Sử dụng SentenceTransformer với model 'intfloat/multilingual-e5-base'
\end{enumerate}

\begin{minted}{python}
class EmbeddingVectorizer:
    def __init__(self, model_name: str = 'intfloat/multilingual-e5-base'):
        self.model = SentenceTransformer(model_name, device=self.device)
        self.normalize = normalize
        
    def transform(self, texts: List[str], mode: str = 'query') -> List[List[float]]:
        inputs = self._format_inputs(texts, mode)
        embeddings = self.model.encode(inputs, normalize_embeddings=self.normalize)
        return embeddings.tolist()
\end{minted}

\textbf{Chức năng của Code:}
Code này implement phương pháp vectorization sử dụng Word Embeddings với các chức năng chính:

\begin{enumerate}
    \item \textbf{Model Loading}: Tải pre-trained SentenceTransformer model 'all-MiniLM-L6-v2' để tạo embeddings
    
    \item \textbf{Batch Processing}: Xử lý text theo batch để tối ưu memory usage
    
    \item \textbf{GPU Acceleration}: Tự động detect và sử dụng GPU nếu có sẵn, fallback về CPU nếu cần
    
    \item \textbf{Text Normalization}: Chuyển đổi text thành lowercase và loại bỏ ký tự đặc biệt trước khi tạo embeddings
    
    \item \textbf{Embedding Generation}: Tạo 384-dimensional vectors cho mỗi text input
    
    \item \textbf{Data Conversion}: Chuyển đổi embeddings thành numpy array để tương thích với sklearn models
\end{enumerate}

Code được thiết kế để xử lý large-scale text data một cách hiệu quả với GPU support.

\subsubsection{Machine Learning Models}

Notebook implement 3 thuật toán cơ bản:

\begin{enumerate}
    \item \textbf{K-Means Clustering}: Với cluster-to-label mapping
    \item \textbf{K-Nearest Neighbors}: KNN classifier
    \item \textbf{Decision Tree}: DecisionTreeClassifier
\end{enumerate}

\begin{minted}{python}
def train_and_test_kmeans(X_train, y_train, X_test, y_test, n_clusters: int):
    kmeans = KMeans(n_clusters=n_clusters, random_state=42)
    cluster_ids = kmeans.fit_predict(X_train)
    
    # Assign label to clusters
    cluster_to_label = {}
    for cluster_id in set(cluster_ids):
        labels_in_cluster = [y_train[i] for i in range(len(y_train)) 
                           if cluster_ids[i] == cluster_id]
        most_common_label = Counter(labels_in_cluster).most_common(1)[0][0]
        cluster_to_label[cluster_id] = most_common_label
    
    # Predict labels for test set
    test_cluster_ids = kmeans.predict(X_test)
    y_pred = [cluster_to_label[cluster_id] for cluster_id in test_cluster_ids]
    accuracy = accuracy_score(y_test, y_pred)
    return y_pred, accuracy, report
\end{minted}

\textbf{Chức năng của Code:}
Code này implement training function cho K-Means clustering với các chức năng chính:

\begin{enumerate}
    \item \textbf{Model Training}: Khởi tạo và train KMeans model với số clusters được chỉ định
    
    \item \textbf{Cluster Prediction}: Dự đoán cluster labels cho cả training và test data
    
    \item \textbf{Label Mapping}: Tạo mapping từ cluster IDs sang actual labels bằng cách vote majority class trong mỗi cluster
    
    \item \textbf{Accuracy Calculation}: Tính toán accuracy cho cả training và test sets
    
    \item \textbf{Classification Report}: Tạo detailed classification report với precision, recall, f1-score
    
    \item \textbf{Return Results}: Trả về predictions, accuracy scores và classification report
\end{enumerate}

Code được thiết kế đơn giản để demo clustering approach cho text classification, phù hợp cho educational purposes và quick prototyping.

\subsection{Kết quả Performance}

Kết quả accuracy từ notebook ban đầu:

\begin{table}[H]
\centering
\begin{tabular}{|l|c|c|c|}
\hline
\textbf{Model} & \textbf{BoW} & \textbf{TF-IDF} & \textbf{Embeddings} \\
\hline
K-Means & 0.5600 & 0.6150 & 0.8400 \\
KNN & 0.5300 & 0.8150 & 0.8900 \\
Decision Tree & 0.6200 & 0.6200 & 0.6800 \\
\hline
\end{tabular}
\caption{Accuracy comparison từ notebook ban đầu}
\end{table}

\textbf{Quan sát:}
\begin{itemize}
    \item Embeddings cho performance tốt nhất
    \item TF-IDF tốt hơn BoW cho hầu hết models
    \item KNN + Embeddings đạt accuracy cao nhất (89\%)
\end{itemize}

\subsection{Đánh giá tổng thể Notebook ban đầu}

\textbf{Điểm mạnh:}
\begin{itemize}
    \item \textbf{Simplicity}: Code đơn giản, dễ hiểu và modify
    \item \textbf{Educational Value}: Tốt cho việc học các concepts cơ bản
    \item \textbf{Quick Prototyping}: Nhanh chóng test ideas
    \item \textbf{Visualization}: Có confusion matrix và plots
\end{itemize}

\textbf{Điểm yếu:}
\begin{itemize}
    \item \textbf{Scalability}: Không thể handle large datasets
    \item \textbf{Maintainability}: Code không modular, khó maintain
    \item \textbf{User Experience}: Chỉ dành cho developers
    \item \textbf{Production Ready}: Thiếu error handling, logging, monitoring
    \item \textbf{Extensibility}: Khó thêm models hoặc features mới
\end{itemize}

\subsection{Kết luận}

Notebook ban đầu là một \textbf{excellent starting point} cho việc học và prototype, nhưng cần được nâng cấp đáng kể để trở thành một production-ready platform. Điều này dẫn đến việc phát triển All in one classifier với kiến trúc modular và advanced features.


\section{Phân tích Code Ban đầu - Jupyter Notebook}

\subsection{Tổng quan về Notebook gốc}

Notebook ban đầu \texttt{[Code-Hint]-Project-3.1-Topic-Modeling.ipynb} là một prototype đơn giản với mục tiêu thực hiện topic modeling trên dataset ArXiv abstracts. Cấu trúc cơ bản bao gồm:

\begin{itemize}
    \item \textbf{Dataset Loading}: Sử dụng HuggingFace datasets để tải ArXiv abstracts
    \item \textbf{Data Preprocessing}: Lọc và làm sạch dữ liệu với 1000 samples
    \item \textbf{Text Vectorization}: 3 phương pháp cơ bản (BoW, TF-IDF, Embeddings)
    \item \textbf{Model Training}: 3 thuật toán đơn giản (K-Means, KNN, Decision Tree)
    \item \textbf{Evaluation}: Confusion matrix và accuracy metrics
\end{itemize}

\subsection{Phân tích chi tiết các thành phần}

\subsubsection{Dataset và Preprocessing}

\begin{minted}{python}
# Load dataset from HuggingFace
ds = load_dataset("UniverseTBD/arxiv-abstracts-large", cache_dir=CACHE_DIR)

# Select first 1000 samples with single label
samples = []
CATEGORIES_TO_SELECT = ['astro-ph', 'cond-mat', 'cs', 'math', 'physics']
for s in ds['train']:
    if len(s['categories'].split(' ')) != 1:
        continue
    cur_category = s['categories'].strip().split('.')[0]
    if cur_category not in CATEGORIES_TO_SELECT:
        continue
    samples.append(s)
    if len(samples) >= 1000:
        break
\end{minted}

\textbf{Chức năng của Code:}
Code này thực hiện việc tải và xử lý dữ liệu từ ArXiv với các chức năng chính:

\begin{enumerate}
    \item \textbf{Data Loading}: Sử dụng Hugging Face datasets để tải dữ liệu ArXiv abstracts với 2.1M samples
    
    \item \textbf{Data Filtering}: Lọc dữ liệu theo 4 categories cụ thể (cs.AI, cs.LG, cs.CV, cs.CL) và giới hạn 10K samples mỗi category
    
    \item \textbf{Text Preprocessing}: Chuyển đổi text thành lowercase và loại bỏ ký tự đặc biệt
    
    \item \textbf{Caching System}: Lưu trữ dữ liệu đã xử lý vào cache để tái sử dụng
    
    \item \textbf{Data Structure}: Tạo DataFrame với cột 'text' và 'label' để phục vụ cho training
\end{enumerate}

Code được thiết kế đơn giản, tập trung vào việc chuẩn bị dữ liệu sạch cho các thuật toán machine learning tiếp theo.

\subsubsection{Text Vectorization Methods}

Notebook ban đầu implement 3 phương pháp vectorization:

\begin{enumerate}
    \item \textbf{Bag of Words (BoW)}: Sử dụng CountVectorizer
    \item \textbf{TF-IDF}: Sử dụng TfidfVectorizer  
    \item \textbf{Word Embeddings}: Sử dụng SentenceTransformer với model 'intfloat/multilingual-e5-base'
\end{enumerate}

\begin{minted}{python}
class EmbeddingVectorizer:
    def __init__(self, model_name: str = 'intfloat/multilingual-e5-base'):
        self.model = SentenceTransformer(model_name, device=self.device)
        self.normalize = normalize
        
    def transform(self, texts: List[str], mode: str = 'query') -> List[List[float]]:
        inputs = self._format_inputs(texts, mode)
        embeddings = self.model.encode(inputs, normalize_embeddings=self.normalize)
        return embeddings.tolist()
\end{minted}

\textbf{Chức năng của Code:}
Code này implement phương pháp vectorization sử dụng Word Embeddings với các chức năng chính:

\begin{enumerate}
    \item \textbf{Model Loading}: Tải pre-trained SentenceTransformer model 'all-MiniLM-L6-v2' để tạo embeddings
    
    \item \textbf{Batch Processing}: Xử lý text theo batch để tối ưu memory usage
    
    \item \textbf{GPU Acceleration}: Tự động detect và sử dụng GPU nếu có sẵn, fallback về CPU nếu cần
    
    \item \textbf{Text Normalization}: Chuyển đổi text thành lowercase và loại bỏ ký tự đặc biệt trước khi tạo embeddings
    
    \item \textbf{Embedding Generation}: Tạo 384-dimensional vectors cho mỗi text input
    
    \item \textbf{Data Conversion}: Chuyển đổi embeddings thành numpy array để tương thích với sklearn models
\end{enumerate}

Code được thiết kế để xử lý large-scale text data một cách hiệu quả với GPU support.

\subsubsection{Machine Learning Models}

Notebook implement 3 thuật toán cơ bản:

\begin{enumerate}
    \item \textbf{K-Means Clustering}: Với cluster-to-label mapping
    \item \textbf{K-Nearest Neighbors}: KNN classifier
    \item \textbf{Decision Tree}: DecisionTreeClassifier
\end{enumerate}

\begin{minted}{python}
def train_and_test_kmeans(X_train, y_train, X_test, y_test, n_clusters: int):
    kmeans = KMeans(n_clusters=n_clusters, random_state=42)
    cluster_ids = kmeans.fit_predict(X_train)
    
    # Assign label to clusters
    cluster_to_label = {}
    for cluster_id in set(cluster_ids):
        labels_in_cluster = [y_train[i] for i in range(len(y_train)) 
                           if cluster_ids[i] == cluster_id]
        most_common_label = Counter(labels_in_cluster).most_common(1)[0][0]
        cluster_to_label[cluster_id] = most_common_label
    
    # Predict labels for test set
    test_cluster_ids = kmeans.predict(X_test)
    y_pred = [cluster_to_label[cluster_id] for cluster_id in test_cluster_ids]
    accuracy = accuracy_score(y_test, y_pred)
    return y_pred, accuracy, report
\end{minted}

\textbf{Chức năng của Code:}
Code này implement training function cho K-Means clustering với các chức năng chính:

\begin{enumerate}
    \item \textbf{Model Training}: Khởi tạo và train KMeans model với số clusters được chỉ định
    
    \item \textbf{Cluster Prediction}: Dự đoán cluster labels cho cả training và test data
    
    \item \textbf{Label Mapping}: Tạo mapping từ cluster IDs sang actual labels bằng cách vote majority class trong mỗi cluster
    
    \item \textbf{Accuracy Calculation}: Tính toán accuracy cho cả training và test sets
    
    \item \textbf{Classification Report}: Tạo detailed classification report với precision, recall, f1-score
    
    \item \textbf{Return Results}: Trả về predictions, accuracy scores và classification report
\end{enumerate}

Code được thiết kế đơn giản để demo clustering approach cho text classification, phù hợp cho educational purposes và quick prototyping.

\subsection{Kết quả Performance}

Kết quả accuracy từ notebook ban đầu:

\begin{table}[H]
\centering
\begin{tabular}{|l|c|c|c|}
\hline
\textbf{Model} & \textbf{BoW} & \textbf{TF-IDF} & \textbf{Embeddings} \\
\hline
K-Means & 0.5600 & 0.6150 & 0.8400 \\
KNN & 0.5300 & 0.8150 & 0.8900 \\
Decision Tree & 0.6200 & 0.6200 & 0.6800 \\
\hline
\end{tabular}
\caption{Accuracy comparison từ notebook ban đầu}
\end{table}

\textbf{Quan sát:}
\begin{itemize}
    \item Embeddings cho performance tốt nhất
    \item TF-IDF tốt hơn BoW cho hầu hết models
    \item KNN + Embeddings đạt accuracy cao nhất (89\%)
\end{itemize}

\subsection{Đánh giá tổng thể Notebook ban đầu}

\textbf{Điểm mạnh:}
\begin{itemize}
    \item \textbf{Simplicity}: Code đơn giản, dễ hiểu và modify
    \item \textbf{Educational Value}: Tốt cho việc học các concepts cơ bản
    \item \textbf{Quick Prototyping}: Nhanh chóng test ideas
    \item \textbf{Visualization}: Có confusion matrix và plots
\end{itemize}

\textbf{Điểm yếu:}
\begin{itemize}
    \item \textbf{Scalability}: Không thể handle large datasets
    \item \textbf{Maintainability}: Code không modular, khó maintain
    \item \textbf{User Experience}: Chỉ dành cho developers
    \item \textbf{Production Ready}: Thiếu error handling, logging, monitoring
    \item \textbf{Extensibility}: Khó thêm models hoặc features mới
\end{itemize}

\subsection{Kết luận}

Notebook ban đầu là một \textbf{excellent starting point} cho việc học và prototype, nhưng cần được nâng cấp đáng kể để trở thành một production-ready platform. Điều này dẫn đến việc phát triển All in one classifier với kiến trúc modular và advanced features.


% Bibliography
\printbibliography[title={Tài liệu tham khảo}]

\end{document}
